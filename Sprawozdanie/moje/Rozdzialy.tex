\chapter{Wprowadzenie}
\section{Streszczenie}
W pracy przedstawiono model opisuj�cy leczenie nowotworu skojarzonymi metodami immunoterapii i chemioterapii. Model ten oparty jest na modelu de Pillis'a oraz modelu Isaeva i Osiopov'a \cite{1}. Uwzgl�dnia on rozw�j kom�rek nowotworowych w organizmie, odpowied� uk�adu immunologicznego -- limfocyt�w naciekaj�cych nowotw�r (TIL), kom�rek naturalnie b�jczych (limfocyt�w NK), limfocyt�w CD8+ -- na nowotw�r oraz leczenie metod� chemioterapii i immunoterapii z u�yciem cytokin: interleukin-2(IL-2) i interferon�w alpha (INF-$\alpha$).
\section{Abstract}
\section{Cel pracy}
\section{Uk�ad pracy}

\chapter{Wst�p}
\section{Nowotwory}
\subsection{Rozw�j nowotwor�w}
\subsection{Rodzaje nowotwor�w}

\section{Sposoby leczenia nowotwor�w}
\subsection{Znaczenie uk�adu immunologicznego}
\subsubsection{Budowa uk�adu immunologicznego}

\subsubsection{Limfocyty T-Cells  CD8+}
\subsubsection{Limfocyty NK}
\subsubsection{tumor infiltrating lymphocytes (TIL)}

\subsection{Immunoterapia}
\subsubsection{interleukins-2 (IL-2)}
\subsubsection{interferon alpha (INF-$\alpha$)}

\subsection{Chemioterapia}
\subsection{Chemioimmunoterapia}

\section{Odniesienie do literatury}
\subsection{Alternatywne modele}

\chapter{Model}
\section{Opis modelu}
\section{Za�o�enia modelu}
\section{R�wnania modelu}

\begin{equation}\label{r_tumor}
\dfrac{dT}{dt}=aT(1-bT)-cNT-DT-K_{T}(1-e^{-M})T-c'TL,
\end{equation}

\begin{equation}\label{r_NKcells}
\dfrac{dN}{dt}=eC-fN+g\dfrac{T^{2}}{h+T^{2}}N-pNT-K_{N}(1-e^{-M})N,
\end{equation}

\begin{equation}\label{r_CD8+Tcells}
\dfrac{dL}{dt}=-mL+j\dfrac{D^{2}T^{2}}{k+D^{2}T^{2}}L-qLT+(r_{1}N+r_{2}C)T-uNL^{2}-K_{L}(1-e^{-M})L+\dfrac{p_{i}LI}{g_{i}}+ \nu_{L}(t),
\end{equation}

\begin{equation}\label{r_cylculLymphocytes}
\dfrac{dC}{dt}=\alpha-\beta C-K_{C}(1-e^{-M})C,
\end{equation}

\begin{equation}\label{r_chemioDrug}
\dfrac{dM}{dt}=-\gamma M+\nu_{M}(t),
\end{equation}

\begin{equation}\label{r_IL-2}
\dfrac{dI}{dt}=-\mu_{i}L-j'LI-k'TI+\nu_{I}(t),
\end{equation}

\begin{equation}\label{r_chemioDrug}
\dfrac{dI_{\alpha}}{dt}=V_{\alpha}(t)-gI_{\alpha},
\end{equation}

gdzie:

\begin{equation}\label{r_chemioDrug}
D=d\dfrac{(\frac{L}{T})^{l}}{s+(\frac{L}{T})^{l}}
\end{equation}

\begin{equation}\label{r_chemioDrug}
c'=c_{CTL}(2-e^{\frac{I_{\alpha}}{I_{\alpha 0}}})
\end{equation}

\section{Parametry modelu}

\chapter{Specyfikacja wewn�trzna}
\chapter{Specyfikacja zewn�trzna}
\chapter{Symulacje}
\section{Scenariusz I}
\section{Scenariusz II}
\section{Scenariusz III}
\section{Scenariusz IV}
\section{Scenariusz V}

\chapter{Rezultaty}
\chapter{Analiza wynik�w}
\chapter{Podsumowanie}




\begin{thebibliography}{9}
\bibitem{1} "Mathematical Model of Cancer Treatments Using Immunotherapy, Chemotherapy and Biochemotherapy", Mustafa Mamat, Subiyanto i Agus Kartono
\bibitem{2} 
\bibitem{3} 
\bibitem{4} 
\bibitem{5} 
\bibitem{6} 
\bibitem{7} 
\end{thebibliography}